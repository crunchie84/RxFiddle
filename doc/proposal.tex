\section{Introduction}%
\label{sec:intro} Reactive Programming has been around for many years,
and recently several implementations have surfaced that are now
incorporated into widely used frameworks and in use for many production
applications.  However, adapting to the paradigm of Reactive Programming
as a developer is known to take some time.  Furthermore, even with basic
or advanced knowledge of an implementation it takes longer to understand
existing Reactive code than simple sequential code.  What can really
help to understand the created data flows are diagrams~\cite{weck2016visualizing}
which show how data is received, transformed and emitted at each step.

The scope of this thesis will be Reactive Extensions (Rx)~\cite{msdn_rx},
one of the libraries implementing Reactive Programming which has
implementations in almost every programming language.  The main
documentation of Rx~\cite{reactivex} uses Marble Diagrams~\cite{c9_marblediagrams}
for each operator to show the behaviour of this single operator.  These
diagrams really complement the name of the operator and its description,
allowing the developer to work-out the nitty details and pick the right
operator for it's use.  They are however only generated per operator,
and are not combined for complete data flows, showing the full flow
through many operators.

In this master's thesis we will focus on creating complete and
interactive Marble Diagrams for full data flows, automatically, from
sources and from running applications.

\subsection{Observable structure analysis} The templates for the data
flows, encapsulated in Observable in Rx, are contained in code.  By
analysing the source code or bytecode these templates can be extracted.
Observables are created by calling several factory methods on the
Observable-class.  After creation they can be passed as variables and
can be transformed by applying operators which generate a new, extended
Observable structure.  Since Observables are (immutable) value types
they can be used multiple times as a basis to create new structures,
therefore possibly creating a tree of related Observable structures.
This structure is the basis for the visualisation.

\subsection{Run-time analysis} By analysing the structure one can know
in advance through which operators possible future data will flow.
During run-time this propagation of data through operators can be
detected.  In Rx the methods onNext, onError and onComplete propagate
data, which can be instrumented to log the invocation to the
visualisation engine.  Every event then gets shown as a marble in the
correct Observable axis.

To instrument code several technologies are available, for Java:  For
example \href{http://asm.ow2.org}{ASM}~\cite{bruneton2002asm,
kuleshov2007using}, which offers very low level bytecode rewriting, or
\href{http://www.eclipse.org/aspectj/}{AspectJ}~\cite{kiczales2001overview}
which leverages AOP~\cite{kiczales1997aspect} to provide a high level
interface to add logic to existing methods.  Either of these libraries
will be used to setup the run-time analysis, depending on which enables
our requirements and is the easiest to implement.

\subsection{Visualisation} The de-facto standard to visualise
Observables is called a Marble Diagram.  The ReactiveX documentation
contains these diagrams, for single operators.  These diagrams originate
from RxJava and are drawn in \href{https://www.omnigroup.com/omnigraffle}
{Omnigraffle}.

While the diagrams in the official documentation are static, some
efforts exist to generate these diagrams automatically.  \href{http://RxMarbles.com}
{RxMarbles.com} is a website which allows the user to drag and reorder
events in for almost all Observable operators, live updating the
corresponding diagram.  \href{https://github.com/jaredly/rxvision}{RxVision}
on the other hand visualises full structures.  It offers a code editor
where one can type JavaScript using RxJs and RxVision will visualise the
structure created in the editor.  RxVision injects code into the RxJS
source which extracts the structure, subscriptions and flowing data.
While RxVision is a great step in the right direction, it does not
integrate into development environments as of September 2016:  it
requires the code to be placed in the online editor.

At some time Microsoft offered a ``Marble Diagram Generator" and \href{http://mnajder.blogspot.nl/2010/03/rxsandbox-v1.html}
{RxSandbox}, which were Windows applications which - looking at Google'd
images - had a catalogus of standard operators and a sandbox to generate
custom diagrams.  However, the source of these tools is not available
and the download links are broken.

\subsection{Generating data}%
\label{sec:gen-data} Testing tools like QuickCheck~\cite{quickcheck}
automate test generation by producing arbitrary input, and by finding
test cases that falsify the test conditions.  When a falsification is
found QuickCheck tries to simplify the test data, pruning data which
does not attribute to the tests failure.  An equally advanced test tool
for data flows would be interesting, but is not in scope.  Generating
data however can be interesting.  Visualising the behaviour of
Observables without running the actual program, based solely on the data
flow structure and generated data could provide valuable insight.
Learning from QuickCheck, reducing to pivotal data can show the various
edge cases of how a data flow can evaluate while keeping the amount of
cases to be considered (and interpreted by humans) at a minimum.

\subsection{Tainting} When looking at the values bubbling through an
Observable structure, values might be produced which are not directly
relatable to their sources.  With pointwise transformations the
developer can trace each output back to a single point of input.
However, operations that fold over time might both use new and reuse
older values.  The relation between these variables might not be clear
over time.  One existing solution to track dependencies between
variables is called tainting~\cite{bell2015dynamic}:  by applying a
taint to a variable, dependent variables either get the same taint or a
mixture of all the taints of it's dependencies.  Implementations of
tainting like Phosphor~\cite{bell2014phosphor} can be evaluated and
might be interesting to integrate.

\section{Research Questions}%
\label{sec:questions} The main research question is:

\begin{quotation}
    \noindent
    Does visualising Reactive Extensions help developers comprehend
    their code and ease the debugging of Observables?
\end{quotation}

\noindent
Several smaller questions must be answered to answer the main question:

\begin{enumerate}
    \item
        Structures:
        \begin{enumerate}
            \item
                Can Observables structures be represented in an abstract
                fashion?
            \item
                Can Observable structures be extracted from source code
                or bytecode?
            \item
                Can run-time behaviour of Observables be extracted such
                that it is appropriate input to a visualizer /
                simulation?
            \item
                Can `smart' (%
                \ref{sec:gen-data}) test input data for Observable
                structures be generated?
        \end{enumerate}

    \item
        Visualisation:
        \begin{enumerate}
            \item
                \label{qstn:marble} Can Marble Diagrams effectively
                convey structures containing more than 1 operator?
        \end{enumerate}

    \item
        Debugger Usability:
        \begin{enumerate}
            \item
                \label{qstn:println} Can our tool (fully) replace
                traditional print-debugging in practice?
            \item
                \label{qstn:autogen} Do developers use automated test
                data in practice?
            \item
                \label{qstn:experience} Does our tool improve the
                development experience when working with Rx?
        \end{enumerate}

\end{enumerate}

\section{Motivation}%
Reactive Programming is a different programming model than sequential
programming.  Traditional development tools are not completely adequate
for this model:  just like Async Programming now has async call stacks
in Chrome, Reactive Programming also needs more tools to make developing
with the technology easier and less painful.

Anecdotal evidence and personal experience suggest developers tend to
just add print and debug statements through the reactive code to get a
sense of the ordering and effects of events over time.  This style of
debugging requires constantly changing the source code, adding and
removing print statements on the go.  When using dynamic languages this
can be a little annoying, but when using statically typed languages the
recompilation can take long and the process slows down the development.

Another approach - which might be preferable to print-debugging - is
creating tests.  However, in large Observable structures there might be
many variation points which would require exponentially many variations
of (minimal) input events to be tested or even to be considered.  After
a bug occurs it might be impossible to know in which state the full
Observable was.  Therefore exploring the different scenarios in a visual
way might help to reproduce the bug.  Especially when giving instant
feedback on how a small change in input changes the output over time, by
using virtual time schedulers.

Sidenote:  the above paragraph also presents another option:  keeping
track of the state (in a smart \& efficient way) and being able to dump
this when something unexpected happens.  One problem with this solution
is that something unexpected can also be the absence of an occurrence,
which is not necessarily detectable.

A reason for scoping this to Rx is how wide Rx is used but mainly how
mature it is.  The implementation of Rx dates back before 2010 and is
very well thought through while newer frameworks like Reactive Streams
and Bacon.js lack these backgrounds.  Rx is very stable and structured,
simplifying the implementation of the prototype.  When completed, it can
then be easily extended to many other Reactive Programming
implementations.

\section{Planning}

\subsection{Scheme} A preliminary planning is defined as:
\begin{table}[h]
    \centering
    \begin{tabular}{@{}ll@{}}
        \textbf{What}               & \textbf{When}           \\ 
        \hline
        Start of project, at Ordina & 12th of September, 2016 \\ 
        Research Proposal ready     & 25th of September, 2016 \\ 
        \hline
        Test prototype 1            & 2nd of December, 2016   \\ 
        User test 1                 & 5th of December, 2016   \\ 
        \hline
        Test prototype 2            & 20th of Januari, 2017   \\ 
        User test 2                 & 23th of Januari, 2017   \\ 
        \hline
        Draft of final report       & 15th of March, 2017     \\ 
        \hline
        Thesis Defense              & 15th of April, 2017     \\ 
    \end{tabular}
\end{table}

\subsection{Contact}

\begin{table}[h]
    \centering
    \begin{tabular}{@{}l@{}}
        \textbf{Student}                \\ 
        \hline
        Herman Banken                   \\ 
        Balthasar van der Polweg, Delft \\ 
        06 - 38 94 37 30                \\ 
        hermanbanken@gmail.com          \\ 
    \end{tabular}
\end{table}

\begin{table}[h]
    \centering
    \begin{tabular}{@{}lll@{}}
        \textbf{Ordina}                 & \textbf{University}     & \textbf{University}     \\ 
        \hline
        Joost de Vries                  & Georgios Gousios        & Prof.dr. H.J.M. Meijer  \\ 
        Ringwade 1, Nieuwegein          & EWI HB08.xxx            & EWI HB08.060 / SV       \\ %06 - 12 89 56 76         & -                   & -                        \\
         
        Joost.de.Vries@ordina.nl        & gousiosg@gmail.com      & H.J.M.Meijer@tudelft.nl \\ 
    \end{tabular}
\end{table}

\subsection{Supervision details} The thesis project will take place
mainly at Ordina, and partly at the Delft Technical University.  Ordina
provides a working place, computer for the thesis, as well as sparring
partners in the form of other students and colleagues of Joost from Code
Star and SMART on the same floor.

\noindent
To discuss the progress several meetings are scheduled:
\begin{enumerate}
    \item
        Weekly meetings with the company supervisor Joost de Vries.
    \item
        Weekly meetings with Georgios Gousios in Delft, or over video
        chat.
    \item
        Bi-Weekly meetings with Erik Meijer over video chat.
\end{enumerate}

\noindent
Furthermore some 'user' (developer) tests will need to be executed,
\begin{itemize}
    \item
        to learn existing workflows;
    \item
        to compare existing workflows to new proposed workflows;
    \item
        to provide input on the usability of the tools;
    \item
        or to measure satisfaction with the new tools
\end{itemize}
for which it would also be very convenient if some employees of Ordina
could volunteer.

\subsection{Risk analysis} The project is subject to several risks,
discussed here.

The first risks are internal to the project.  The scope described in
\autoref{sec:intro} and~%
\ref{sec:questions} is quite challenging.  The visualisation part could
be a thesis topic on it's own.  However, due to previous and available
work in projects like RxVision and RxMarbles, the time required for
implementation is at least limited.  The existing visualisation of
RxVision might prove to be not ideal, and RxMarbles is not as complete
as RxVision, so some additional work might be required to create an
optimal visualisation.  This would be perfecting the tool however, and
does not need be part of the academic thesis.

Secondly the project needs a case study and user test to fully test the
effectiveness of the debug methodology.  User studies are a risk since
the organisation of the test event depends on many people.  By doing the
thesis at Ordina this risk is at least limited, as there are many
developers present, on location, of which only a subset needs to be
available.

External risks are other courses that need to be finished.  As of
September 2016 only 7 ECTS need to be completed, not regarding the 45
ECTS of the thesis itself.  I'm currently still working on `IN4306
Literature Study' (10 ECTS) on a somewhat related but more general
subject of `Reactive Programming'.  The remaining work is limited, but -
at the very latest - needs to be completed before the defense.  The
literature study will not take up time from the thesis, as I plan to do
this in the weekend and evenings.

Finally, a risk is the time of the people involved, especially professor
Meijer.  Meijer works a full-time job at Facebook, as of September 2016,
and his professorship is only part-time.  To remedy this risk an
additional supervisor in the person of Georgios Gousios was contacted.
Georgios will function as the default university contact, while Meijer
will provide valuable input where possible.

\subsection{User tests and prototypes} The research question in general,
and specifically subquestions%
\ref{qstn:marble},%
\ref{qstn:println},%
\ref{qstn:autogen} and%
\ref{qstn:experience} touch the man-machine-interaction and psychology
sides of Computer Science.  The appropriate way to answer these
questions would be (one or more) case studies and user tests.

The planning mentions the completion of two prototypes and subsequent
user tests.  The final feature-set of these prototypes can not yet be
determined, but a preliminary specification is given here.

\subsubsection{User tests} To evaluate questions%
\ref{qstn:println} and%
\ref{qstn:experience} the tool needs to be working on all levels of the
implementation.  Both the gathering of data as visualisation need to
work.  Not every feature of Rx needs to be supported, but to test the
experience at least common use cases - that developers can relate to -
should be fully debuggable.  As this requires the bulk of work, these
questions will be addressed in the second test.

Question%
\ref{qstn:marble} can be tested using the visualisation part only.
Building on the existing \href{http://rxmarbles.com}{RxMarbles.com} the
visualisation will be created for some scripted examples.  A user test
can then verify that the visualisation is comprehensible and clear.  To
test the debugging usability of the visualisation a bug can be
introduced in code, and the corresponding visualisation should then be
used to localise the bug.  The visualisation part is the only
requirement for the test, so this question will be addressed in the
first test.

The last question%
\ref{qstn:autogen} is self-contained, and builds upon the visualiser.
Depending on the progress made, the implementation of auto-generation of
tests can be considered or postponed.  Preferably this feature would be
part of prototype 1, to better distribute the tests.

\subsubsection{Prototypes} The list of features for the prototypes then
becomes:

\begin{enumerate}
    \item[Prototype 1]
        \begin{enumerate}
            \item
                Visualiser for Observable sequence with multiple
                subsequent operations
            \item
                Interactive input sequences
            \item
                Live updating events in subsequent sequences
            \item
                Optionally, `smart' test event generation
        \end{enumerate}
    \item[Prototype 2]
        \begin{enumerate}
            \item
                Static analysis collector for structures in code
            \item
                Runtime analysis collector for events in Observable
                sequences
            \item
                Interface to select `root' Observable:  which Observable
                to use as starting point for the visualisation
            \item
                Interface to switch between runtime events and (interactive/generated)
                test events
            \item
                `Smart' test event generation, if not done in prototype
                1.
        \end{enumerate}
\end{enumerate}

\bibliography{papers/references}
{}
\bibliographystyle{plain}
